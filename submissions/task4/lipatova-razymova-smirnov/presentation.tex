\documentclass{beamer}
\usepackage{xmpmulti}
\usepackage[russian]{babel}
\title{Отчет по заданию 4
\newline
Разведовательный анализ данных}
\author{«МОСКОВСКИЙ ГОСУДАРСТВЕННЫЙ УНИВЕРСИТЕТ» ИМЕНИ М.В. ЛОМОНОСОВА}
\date[]{Липатова Александра, Разумова Вера, Смирнов Михаил}

\begin{document}

\begin{frame}
  \titlepage
\end{frame}

\begin{frame}{Постановка задачи}

\begin{center}
    \textbf{Необходимо провести разведывательный анализ данных с целью ответа на вопрос: "С каким из поставщиков стали следует заключить договор?". То есть задача сводится к транспортной задаче по поиску максимально выгодного поставщика.}
\end{center}
\begin{figure}[h]
\centering
\includegraphics[width=0.5\linewidth]{visual16.png}
\label{fig:mpr}
\end{figure}
\end{frame}

\begin {frame}{Основные понятия}

\textbf{Разведочный анализ данных (англ. exploratory data analysis, EDA)} — анализ основных свойств данных, нахождение в них общих закономерностей, распределений и аномалий, построение начальных моделей, зачастую с использованием инструментов визуализации.
\newline

Понятие введено математиком Джоном Тьюки, который сформулировал цели такого анализа следующим образом:
\newline
\textf{• максимальное «проникновение» в данные}
\newline
\textf{• выявление основных структур}
\newline
\textf{• выбор наиболее важных переменных}
\newline
\textf{• обнаружение отклонений и аномалий}
\newline
\textf{• проверка основных гипотез}
\newline
\textf{• разработка начальных моделей.}
\newline

\end{frame}

\begin{frame}{Ход работы}

\textf{1. Получить данные из CSV таблицы.}
\newline

\textf{2. Провести анализ количества продукции, произведенного компаниями Harpy и Westeros каждый месяц.}
\newline

\textf{3. Для большей наглядности изобразим разницу количества дефектов на графике.}
\newline

\textf{4. Рассмотреть плотность распределения усредненных по месяцам дефектов к произведенному товару.}
\newline

\textf{5. Построим графики количества дефектов продукции компании для каждого месяца.}

\end{frame}

\begin{frame}{Описание программы}
\textbf{1. Построим диаграмму количества продукции по месяцам.}
\begin{figure}[h]
\centering
\includegraphics[width=0.6\linewidth]{visual10.png}
\label{fig:mpr}
\end{figure}
\end{frame}

\begin{frame}{Описание программы}
\textbf{2. Построим диаграмму количества дефектов по месяцам.}
\begin{figure}[h]
\centering
\includegraphics[width=0.6\linewidth]{visual11.png}
\label{fig:mpr}
\end{figure}
\end{frame}


\begin{frame}{Описание программы}
\textbf{3. Построим график количества дефектов для более наглядной картины.}
\begin{figure}[h]
\centering
\includegraphics[width=0.9\linewidth]{visual12.png}
\label{fig:mpr}
\end{figure}
\end{frame}

\begin{frame}{Описание программы}
\textbf{4. Построим диаграмму плотности распределения усредненных по месяцам дефектов к произведенному товару}
\begin{figure}[h]
\centering
\includegraphics[width=0.9\linewidth]{visual13.png}
\label{fig:mpr}
\end{figure}
\end{frame}

\begin{frame}{Описание программы}
\textbf{5. Построим графики количества дефектов по всем месяцам.}
\begin{figure}[h]
\centering
\includegraphics[width=0.5\linewidth]{visual15.png}
\label{fig:mpr}
\end{figure}
\end{frame}

\begin{frame}{Выводы}
\textf{ Видно, что при примерно одинаковом количестве произведенной продукции среднее количество дефектов компании Harpy ниже, чем компании Westeros. На основании проделанного анализа делаем вывод, что работать стоит с компанией Harpy.}
\begin{figure}[h]
\centering
\includegraphics[width=0.4\linewidth]{visual19.png}
\label{fig:mpr}
\end{figure}
\end{frame}



\end{document}
