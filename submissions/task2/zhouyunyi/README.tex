\documentclass[12pt,a4paper]{paper}
\usepackage[utf8]{inputenc}
\usepackage[T2A]{fontenc}
\usepackage{amsmath}
\usepackage{amsfonts}
\usepackage{amssymb}
\usepackage{listings}
\usepackage{pgfplots}
\usepackage{tikz}
\usetikzlibrary{positioning}-
\lstset{language=C,basicstyle=\ttfamily,tabsize=4}
\usepackage[russian]{babel}
\usepackage[left = 2cm, right = 2cm, bottom = 2cm]{geometry}
\begin{document}
\begin{center}
\large\bfseries Московский Государственный Университет \\
\bfseries\large имени М.В.Ломоносова\\
\bfseries\large факультет \\
\bfseries\large Вычислительной Математики и Кибернетики
\end{center}
\vspace{1mm}
\begin{figure}[h]
\centering
\includegraphics[width=0.2\textwidth]{logo-mgu}
\end{figure}
\vspace{2mm}
\vspace{40mm}
\begin{center}
\bfseries\Large\bf Практическое задание 
\textnumero 2\\
\bfseries\large выполнено студентами: Ли Олег, Сиверина Анастасия, Юньи Чжоу
\end{center}
\vspace{50mm}
\vspace{30mm}
\begin{center}
\bfseries\large Москва\\ 2019
\end{center}
\pagestyle{empty}
\newpage
\bfseries{Существуют две основные цели анализа временных рядов:} \mdseries\\ 

1)Определение природы ряда.\\ 

2)Прогнозирование (предсказание будущих значений временного ряда по настоящим и прошлым значениям).\\

Обе эти цели требуют, чтобы модель ряда была более или менее, формально описана.\\
 
\bfseries{Что такое временной ряд?} \mdseries\\
- это последовательность значений, описывающих протекающий во времени процесс, измеренных в последовательные моменты времени, обычно через равные промежутки.\\

\hline  \hline
\begin{center}
\bfseries\large{1 этап:}\\
\end{center}

\bfseries{Тест Дики - Фуллера} \mdseries - это методика, которая используется в прикладной статистике и эконометрике для анализа временных рядов для проверки на стационарность. Является одним из тестов на единичные корни. Был предложен в 1979 году Дэвидом Дики и Уэйном Фуллером.\\

Временной ряд имеет единичный корень, или порядок интеграции один, если его первые разности образуют стационарный ряд.\\
Это условие записывается как $y_t=I(1)$, если ряд первых разностей $\triangle y_t = y_t-y _{t-1}$ является стационарным $\triangle y_t=I(0)$.\\

При помощи этого теста проверяют значение коэффициента $a$ в авторегрессионном уравнении первого порядка AR(1).\\

$$y_t =a\cdot y _{t-1}+ \varepsilon_t,$$

где $y_{t}$ - временной ряд, а $\varepsilon$ - ошибка. \\

Если $a=1$, то процесс имеет единичный корень, в этом случае ряд $y_{t}$ не стационарен, является интегрированным временным рядом первого порядка — $I(1)$.\\

Если $|a|<1$, то ряд стационарный - $I(0)$. \\

Ряд является \bfseries{стационарным} \mdseries, если он совершает колебания вокруг своего математического ожидания.\\

Если первые разности ряда стационарны, то он называется \bfseries{инегрируемым рядом первого порядка} \mdseries. 
\newpage


\bfseries{Сущность DF-теста} \mdseries\\

Приведенное авторегрессионное уравнение AR(1) можно переписать в виде:\\
$$\triangle y_{t}=b\cdot y_{{t-1}}+\varepsilon _{t},$$\\
 
$b=a-1$,\\

$\triangle$  - оператор разности первого порядка $\triangle y_{t}=y_{t}-y_{t-1}$.\\

Поэтому проверка гипотезы о единичном корне в данном представлении означает проверку нулевой гипотезы о равенстве нулю коэффициента $b$. \\

Статистика теста (DF-статистика) — это обычная t-статистика для проверки значимости коэффициентов линейной регрессии. Однако, распределение данной статистики отличается от классического распределения t-статистики (распределение Стьюдента или асимптотическое нормальное распределение). Распределение DF-статистики выражается через винеровский процесс и называется распределением Дики - Фуллера. \\

\bfseries{Существует три версии теста (тестовых регрессий):} \mdseries\\

1) Без константы и тренда\\

    $$\triangle y_{t}=b\cdot y_{{t-1}}+\varepsilon _{t}.$$

2)С константой, но без тренда:\\

    $$\triangle y_{t}=b_{0}+b\cdot y_{{t-1}}+\varepsilon _{t}.$$

3)С константой и линейным трендом:\\

    $$\triangle y_{t}=b_{0}+b_{1}\cdot t+b\cdot y_{{t-1}}+\varepsilon _{t}.$$
    
    
Для каждой из трёх тестовых регрессий существуют свои критические значения DF-статистики, которые берутся из специальной таблицы Дики - Фуллера. Если значение статистики лежит левее критического значения (критические значения - отрицательные) при данном уровне значимости, то нулевая гипотеза о единичном корне отклоняется и процесс признается стационарным. Иначе гипотеза не отвергается и процесс может содержать единичные корни, то есть быть нестационарным (интегрированным) временным рядом.\\
\hline  \hline


\newpage


\hline  \hline
\begin{center}
\bfseries\large{2 этап:}\\
\end{center}


\bfseries{Тренд:} \mdseries - линейная составляющая плюс случайная составляющая, которая является стационарным временным рядом с нулевым средним.\\

Кроме линейного встречаются также: квадратичный, экспоненциальный и т.д. тренды.\\

\bfseries{Тренд:} \mdseries- некоторая тенденция к изменению значений временного ряда.\\

\bfseries{Каким может быть тренд?} \mdseries\\

1)Восходящим(рынок растет)\\

2)Нисходящий(рынок падает)\\

3)Тренд отсутствует\\

Чтобы выделить тренд, для начала нужно убедиться в том, что тренд в данных у нас действительно значим. То есть мы для начала должны проверить статистическую гипотезу следующего вида:\\

$H_0: m_i=m, i=1,2,...,n$\\

То есть мы предполагаем, что каждое наблюдение в нашем временном ряде - это некоторая случайная величина, и мы проверяем гипотезу о том, что все эти случайные величины имеют одинаковое математическое ожидание. То есть наша нулевая гипотеза формулируется как гипотеза об отсутствии тренда в данных. И мы проверяем ее против альтернативной гипотезы следующего вида:\\

$H_1: |m_{i+1}-m_i|>0, i=1,2,...,n-1$\\

\bfseries{Сезонные компоненты} \mdseries - некоторые периодически повторяющиеся явления в данных.

Сезонные компоненты выделяются для того, чтобы использовать их при прогнозе. То есть если мы хотим не только проанализировать те данные, которые у нас есть, но и хотим попробовать спрогнозировать наш ряд на будущие периоды, то, соответственно, мы должны учитывать, что на эти значения в будущем также будет влиять сезонность.\\

\newpage

\bfseries{Каким образом обычно выделяют сезонные компоненты?} \mdseries\\

На самом деле достаточно часто визуально очень трудно определить по ряду, содержит ли он ту или иную сезонность или нет. Поэтому на практике обычно используют такое понятие, как автокорреляционная функция.\\

Автокорреляционная функция у нас имеет следующий вид:\\
$$ acf_k = \frac{\sum_{i=k+1}^{n}(x_i-x^k)(x_{i-k}-x^{k+1})}{(\sum_{i=k+1}^{n}(x_i-x^k)^2 \sum_{i=k+1}^{n}(x_{i-k}-x^{k+1})^2)^{0,5}}$$
$$x^k=\frac{1}{n-k}\sum_{i=k+1}^{n}x_i$$
$$x^{k+1}=\frac{1}{n-k}\sum_{i=k+1}^{n}x_{i-k}$$

\bfseries{Что такое коэффициент автокорреляции?} \mdseries\\

- величина, которая проверяет наличие зависимости между значениями в выборке.\\

Коэффициент автокорреляции первого порядка считается $с-1$. Он проверяет наличие зависимости между соседними значениями. Говорит нам о том, что мы брали подвыборки со сдвигом на $1$.
Ну и мы можем посчитать коэффициент автокорреляции любого порядка со сдвигом на любую величину. Мы считаем коэффициент автокорреляции для различных сдвигов и затем представляем эти данные графически.\\

Всегда важно помнить про то, что перед определением сезонности нужно вычесть тренд из наших данных.\\

Модели, где временной ряд представлен в виде суммы трендовой, сезонной и случайной компонентов, называются \bfseries{аддитивными} \mdseries, если в виде произведения- \bfseries{мультипликативными моделями.} \mdseries\\

Аддитивная модель имеет вид: $Y=T+S+E$\\

Мультипликативная модель имеет вид: $Y=T\cdot S \cdot E$\\
\hline  \hline


\newpage

\hline  \hline
\begin{center}
\bfseries\large{3 этап:}\\
\end{center}


\bfseries{Авторегрессионная (AR)- модель} \mdseries\\ - модель временных рядов, в которой значения временного ряда в данный момент линейно зависят от предыдущих значений этого же ряда.\\

\bfseries{МА-модель} \mdseries\\ - то есть просто скользящее среднее, модель, в которой значения функции каждой точки равны среднему значению исходной функции за предыдущий период.
И вполне кажется логичным взять и объединить эти две модели. Действительно, так было сделано. Взяли модели \bfseries{AR} \mdseries и \bfseries{MA} \mdseries и объединили. Получили модель \bfseries{ARMA} \mdseries. Все логично. \\

Модель \bfseries{ARMA} \mdseries обозначается как $ARMA(p, q)$, где $p$ и $q$ это целые числа, и $p$ определяет порядок регресии, а $q$ определяет порядок скользящего среднего.\\ 

В общем виде \bfseries{ARMA} \mdseries выглядит следующим образом:

$$X_t=c+\varepsilon_{t}+\sum_{i=1}^{p}a_t \cdot X_{t-i}+\sum_{i=1}^{q}b \cdot \varepsilon_{t-i},$$

где $a_1,..., a_p$ - параметры модели (коэффициенты авторегресии),\\ 

$b_1, .., b_q$ -  параметры модели скользящего среднего,\\ 

$c$ - некоторая постоянная, \\

$\varepsilon_{t}$ - белый шум. \\

\bfseries{ARMA} \mdseries-модель может интерпретироваться как линейная модель множественной регрессии, в которой в качестве объясняющих переменных выступают прошлые значения самой модели, а в качестве регрессионных остатков - скользящее среднее из элементов белого шума. \\

\bfseries{ARMA} \mdseries-модель применима только для стационарных рядов.\\

\bfseries{ARIMA} \mdseries-модель позволяет решать проблему с нестационарными рядами.\\

На самом деле модель \bfseries{ARIMA} \mdseries - это то же самое, что \bfseries{ARMA} \mdseries, только рассматриваются здесь приращения.\\

Модель $ARIMA(p,d,q)$ означает, что разности временного ряда порядка $d$ подчиняются модели $ARMA(p,q)$.\\

$$\triangle^d X_t=c+\varepsilon_{t}+\sum_{i=1}^{p}a_i\triangle^d X_{t-i}+\sum_{i=1}^{q}b_i\varepsilon_{t-i},$$ 

где $\triangle^d$ - оператор разности временного ряда порядка $d$,\\

$c, a_i, b_i$ - параметры модели.\\





\end{document}
