\documentclass[12pt,a4paper]{paper}
\usepackage[utf8]{inputenc}
\usepackage[T2A]{fontenc}
\usepackage{amsmath}
\usepackage{amsfonts}
\usepackage{amssymb}
\usepackage{listings}
\usepackage{pgfplots}
\usepackage{tikz}
\usetikzlibrary{positioning}
\lstset{language=C,basicstyle=\ttfamily,tabsize=4}
\usepackage[russian]{babel}
\usepackage[left = 2cm, right = 2cm, bottom = 2cm]{geometry}
\begin{document}
\begin{center}
\bfseries\large{Практическое задание №1 выполнено студентами: Ли Олег, Сиверина Анастасия, Юньи Чжоу}\\
\bfseries{Для реализации данного задания, мы выбрали метод итераций (метод Брауна-Робинсон)}\\
\end{center}

Но перед тем, как решать игру, нужно, прежде всего, попытаться ее упростить, избавившись от лишних стратегий.

В общем виде матричная игра может быть записана следующей платежной матрицей:
$$
\begin{tabular}{|l|l|l|l|l|}
  \hline
  i | j & B_1 & B_2 & \dots & B_n   \\ \hline
   A_1 & a_{1,1} & a_{1,2} & \dots & a_{1,n}   \\ \hline
   A_2 & a_{2,1} & a_{2,2} & \dots & a_{2,n}   \\ \hline
    \dots & \dots & \dots & \dots & \dots   \\ \hline
   A_m & a_{m,1} & a_{m,2} & \dots & a_{m,n}   \\ \hline
  \hline
\end{tabular}
$$
  \hline  \hline
\begin{center}
\bfseries\large{1 этап:}\\
\end{center}
Для этого введем понятие \textbf{доминирования.}\\
Стратегия $A_i$ игрока $А$ называется доминирующей над стратегией $A_k$, если в строке $A_i$ стоят выигрыши не меньшие, чем в соответствующих клетках строки $A_k$, и из них по крайней мере один действительно больше, чем в соответствующей клетке строки $A_k$. Если все выигрыши строки $A_i$ равны соответствующим выигрышам строки $A_k$, то стратегия $A_i$ называется \bfseries{дублирующей} \mdseries стратегию $A_k$.

В нашей программе данную задачу реализует функция $func1$, которая при помощи встроенной функции $roll$ циклически сдвигает строки нашей матрицы. Эту матрицу мы отнимаем от нашей исходной матрицы. Если все элементы данной строки больше либо равны 0, то данную строку удаляем. 
\hline

Аналогично определяются доминирование и дублирование для стратегий игрока $В$: доминирующей называется та его стратегия, при которой везде стоят выигрыши не большие, чем в соответствующих клетках другой, и по крайней мере один из них действительно меньше; дублирование означает полное повторение одного столбца другим. Естественно, что если для какой-то стратегии есть доминирующая, то эту стратегию можно отбросить; также отбрасываются и дублирующие стратегии. 

В нашей программе данную задачу реализует функция $func2$, которая при помощи встроенной функции $roll$ циклически сдвигает столбцы нашей матрицы. Эту матрицу мы отнимаем от нашей исходной матрицы. Если все элементы данного столбца меньше либо равны 0, то данный столбец удаляем. 

  \hline  \hline
\begin{center}
\bfseries\large{2 этап:}\\
\end{center}

Далее исследуем наличие седловой точки. \bfseries{Если матричная игра имеет седловую точку, то верхняя и нижняя цены матричной игры одинаковы.} \mdseries 

Для этого выписываем минимумы строк и из них выбираем наибольший. Это \bfseries{нижняя цена игры} \mdseries, или еще называют \bfseries{максимином.} \mdseries
$$v = max min a_{i,j}$$

Затем выписываем максимумы столбцов и из них выбираем наименьший. Это \bfseries{верхняя цена игры} \mdseries, или \bfseries{минимакс.} \mdseries
$$v = min max a_{i,j}$$

Таким образом, если соответствующий элемент одновременно является наименьшим в строке и наибольшим в столбце и равен цене игры, то существует седловая точка.

Стратегии $A_i,B_j$, при которых этот выигрыш достигается, называются \bfseries{оптимальными чистыми стратегиями} \mdseries, а их совокупность - \bfseries{решением игры.} \mdseries Про саму игру в этом случае мы утверждаем, что она решается в чистых стратегиях. Обеим сторонам $А$ и $В$ можно указать их оптимальные стратегии, при которых их положение - наилучшее из возможных.

Данную задачу реализует функция $sedlo$, которая выбирает минимумы строк и максимумы столбцов, из которых после выбирается максимальный и минимальный элемент соответственно. Если существует такой элемент, который является наименьшим в строке и наибольшим в столбце, тозадача решается в чистых стратегиях и оптимальные стратегии: $$p = {0, \dots, 0, p_i, 0, \dots, 0}$$
$$q = {0, \dots, 0, q_j, 0, \dots, 0} $$
где $p_i =  q_j = 1$,\\
$i = 1, \dots, n $,\\
$j = 1, \dots, m $\\

Если же седловой точки не существует, то переходим к следующему этапу.
  \hline  \hline

\begin{center}
\bfseries\large{3 этап:}\\
\end{center}
Но седловая точка в игре может и не существовать. В этом случае мы воспользуемся \bfseries{методом итераций.} \mdseries

Идея его в следующем: разыгрывается «мысленный эксперимент», в котором стороны $А$ и $В$ поочередно применяют друг против друга свои стратегии, стремясь выиграть побольше (проиграть поменьше).

Начнем с того, что один из игроков (скажем, $A$ ) выбирает произвольно одну из своих стратегий $A_i$. Противник ($B$) отвечает ему той из своих стратегий $B_j$, которая хуже всего для $A$. Дальее снова очередь $А$, затем очередь $B$. И так далее: на каждом шаге итерационного процесса каждый игрок отвечает на очередной ход другого той своей стратегией, которая является оптимальной для него относительно смешанной стратегии другого, в которую все примененные до сих, пор стратегии входят пропорционально частотам их применения. 

Вместо того, чтобы вычислять каждый раз средний выигрыш, можно пользоваться просто «накопленном» за предыдущие ходы выигрышем и выбирать ту свою стратегию, при которой этот накопленный выигрыш максимален (минимален). 

\bfseries{Доказано, что такой метод сходится: при увеличении числа «партий» средний выигрыш на одну партию будет стремиться к цене игры, а частоты применения стратегий — к их вероятностям. в оптимальных смешанных стратегиях игроков.} \mdseries


Данную задачу реализует функция $nash_equilibrium$, 
Массивы $summ1$ и $summ2$ созданы для поиска минимального  и максимального значения наших стратегий. Для начала берем любую стратегию и прибавляем ее к $summ1$, ищем минимальный элемент и определяем на какой позиции он находится. Далее все делаем аналогично, только с поиском максимального значения $summ2$.
Остается посчитать только цену игры.
$$v = \frac{MIN}{c}$$
$$V = \frac{MAX}{c}$$
А наша цена игры- это среднее арифметическое между $v$ и $V$.
\hline  \hline
Для запуска программы необходимо в терменале определить размер матрицы(n m), и задать саму матрицу.\\  
\hline  \hline
Задача №1 была разделена на этапы:\\

1 этап был реализован Сивериной Анастасией\\

2 этап был реализован Юньи Чжоу\\

3 этап был реализован Ли Олегом\\

Задача №2 была реализована Сивериной Анастасией\\

Задача №3 была реализована Юньи Чжоу\\

Задача №4 была реализована Ли Олегом\\

\end{document}
