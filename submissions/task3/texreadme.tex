\documentclass{article}
\usepackage[utf8]{inputenc}
\usepackage[english,russian]{babel}
\title{README

Applepen.}
\date{Весенний семестр 2020 г.}

\usepackage{natbib}
\usepackage{graphicx}
\usepackage{xcolor}
\usepackage{hyperref}
\definecolor{linkcolor}{HTML}{0067A5} % цвет ссылок
\definecolor{urlcolor}{HTML}{0067A5} % цвет гиперссылок
\hypersetup{pdfstartview=FitH,  linkcolor=linkcolor,urlcolor=urlcolor, colorlinks=true}
\begin{document}

\maketitle

\section{Постановка задачи.}
Applepen - большая торговая сеть, которая занимается продажей всего двух продуктов: яблок и карандашей. Её магазины расположены в различных уголках Соединённых Штатов и более 10 лет обслуживает покупателей.
\newlineВ CSV-файлах дана информация, отсортированная по дате:
\begin{enumerate} 
\item о закупках (поставки яблок и карандашей два раза в месяц),
\item о продажах (лог транзакций, по записи на каждую проданную позицию),
\item об инвентаре (месячные данные общего количества яблок и карандашей на складе).
\end{enumerate} 
Необходимо получить следующие данные в CSV-файлах:
\begin{enumerate} 
\item состояние склада на каждый день,
\item месячные данные о количестве сворованного товара,
\item агрегированные данные об объёмах продаж и количестве сворованной продукции по штату и году.
\end{enumerate}
Наборы входных и выходных данных для тестирования решения скачаны по ссылке: \href{https://console.cloud.google.com/storage/browser/artem-pyanykh-cmc-prac-task3-seed17}
{https://console.cloud.google.com/storage/browser/artem-pyanykh-cmc-prac-task3-seed17}.
В скачанной папке с исходными данными содержится 3 типа файлов для каждого магазина в штате:
\begin{enumerate} 
\item \textbf{(имя штата)-(название магазина)-supply.csv} - информация по поставкам (1го и 15го числа месяца),
\item \textbf{(имя штата)-(название магазина)-sell.csv} - транзакции на каждый проданный продукт,
\item \textbf{(имя штата)-(название магазина)-inventory.csv} - месячные данные общего количества 
яблок и карандашей на складе.
\end{enumerate}
\section{Математическое решение задачи.}
Топ-менеджеры требуют предоставить информацию о состоянии склада на каждый день.  
Мы знаем, что поступления на склад происходят только два раза в месяц – 1 и 15 числа, 
так же нам известно о всех транзакциях за день и о количестве товара на складе в конце каждого месяца. Рассмотрим модель относительно одного товара, так как яблоки и карандаши независимы, 
вычисления будут одинаковыми. 
\newline Представим, что у нас есть данные, что сегодня на складе находится \newline\textbf{СКЛАД} единиц товара, просуммировав все транзакции за день, 
мы точно знаем, что сегодня всего было распродано \textbf{ПРОДАЛИ} единиц товара. 
Тогда завтра на складе будет: 
\begin{center}
\textbf{СКЛАД\underline{ }ЗАВТРА = СКЛАД – ПРОДАЛИ}
\end{center}
Так мы вычисляем данные до дня поставок. 
В день, когда завозят новый товар, на складе становится на ПОСТАВКА единиц товара больше, 
поэтому состояние склада будет определяться, как 
\begin{center}
\textbf{СКЛАД\underline{ }ЗАВТРА = СКЛАД + ПОСТАВКА – ПРОДАЛИ}
\end{center}
При этом мы знаем, что сотрудники воруют: 
количество реального товара и количество товара “на бумаге” отличается, 
поэтому в конце каждого месяца в файл \textbf{daily}, 
в котором мы храним информацию о состоянии склада на каждый день, 
записываем данные из \textbf{inventory} – действительное количество товара. 
На этом шаге, мы параллельно заполняем и файл \textbf{steal}, 
в котором хранится месячная информация о ворованном товаре, записывая туда разницу между значениями 
\begin{center}
\textbf{УКРАЛИ = СКЛАД\underline{ }РЕЗУЛЬТАТ - СКЛАД\underline{ }РЕАЛЬНОСТЬ}
\end{center}
Первое из них – это то, что мы получили по информации о поставках и продажах, 
а второе - после инвентаризации \textbf{СКЛАД\underline{ }РЕАЛЬНОСТЬ}. 
\newline А что касательно последнего пункта? Агрегация, или агрегирование 
(лат. aggregatio "присоединение") — процесс объединения элементов в одну систему. 
То есть теперь, нам надо объединить полученные нами данные по году и штату, 
в котором расположены торговые точки. 
\section{Программный подход к решению задачи.}
Своё решение мы представим в виде программы на языке Python. Нам понадобятся:
\begin{itemize}
  \item модуль \textbf{pandas} – библиотека для обработки и анализа данных, 
работы с объектами DataFrame, метод pandas.testing.assert\underline{ }frame\underline{ }equal 
– для сравнения двух DateFrame;
\item класс \textbf{datetime.timedelta} – для вычисления  разница между двумя моментами времени, 
с точностью до микросекунд;
\item класс \textbf{datetime.datetime} - содержит информацию о времени и дате, 
основываясь на данных из Григорианского календаря;
\item модуль \textbf{оs} - библиотека функций для работы с операционной системой. 
Методы, включенные в неё позволяют определять тип операционной системы, 
получать доступ к переменным окружения, управлять директориями и файлами. 
Например, os.listdir - список файлов и директорий в папке, 
os.mkdir  - создаёт директорию, os.path.isfile - является ли путь файлом, 
os.path.join - соединяет пути с учётом особенностей операционной системы;
\item модуль \textbf{shutil} содержит набор функций высокого уровня для обработки файлов, групп файлов, и папок. В частности, shutil.rmtree - удаляет текущую директорию и все поддиректории. 
      \end{itemize}
Отметим, что в реализуемой нами программе важно, 
чтобы входные данные хранились в папке \textbf{input}, выходные – в \textbf{answer}, 
при этом они находились в одной директории вместе с \textbf{Task\underline{ }3.ipynb}. 
\newlineРезультатом выполнения программы будет папка \textbf{our\underline{ }answer}, 
созданная в той же директории, в которой находится сама программа и входные данные. 
В эту папку записываются результаты в следующем формате:
\begin{enumerate} 
\item \textbf{(имя штата)-(название магазина)-daily.csv} - состояние склада на каждый день,
\item \textbf{(имя штата)-(название магазина)-steal.csv} - месячные данные о количестве сворованного товара,
\item \textbf{aggregation} - агрегированные данные об объемах продаж и количестве сворованной продукции по штату и году.
\end{enumerate}
По ходу программы мы рассматриваем магазины, которые есть в папке \textbf{input}, 
каждый раз вызывая функцию \textbf{shop\underline{ }processing(inventory, sell, supply, state)}, 
параметрами которой являются считанные файлы и название штата, 
а результатом - \textbf{daily, steal, aggregation}. 
\newline Первые два файла строятся по описанному выше алгоритму, 
для последнего в процессе работы мы считаем \textbf{apple\underline{ }sold\underline{ }year, pen\underline{ }sold\underline{ }year, apple\underline{ }stolen, pen\underline{ }stolen} простым суммированием.
Чтобы получить исходные агрегированные данные об объемах продаж и количестве сворованной продукции 
по штату и году, мы сначала конкатенируем все полученные таблицы aggregation  с помощью метода contact,
затем группируем их по штату и году с помощью метода groupby и для завершения суммируем значения по полям.
\newline В итоге получается такая формула:

\textbf{aggregation\underline{ }sum = pd.concat([aggregation\underline{ }sum,aggregation]).}

\textbf{groupby(['year','state'])}

\textbf{['apple\underline{ }sold', 'apple\underline{ }stolen', 'pen\underline{ }sold', 'pen\underline{ }stolen'].sum()}
\newlineДля завершения работы, чтобы проверить корректность полученных нами данных, 
мы сверяем файлы из двух папок answer и our\underline{ }answer, 
если встречаем различия, то выбрасываем исключение и выводим соответствующую информацию. 
\newline \newline\textbf{Работу выполнили Корнюхина Софья, Попова Диана, Ситникова Екатерина, 
311 группа; совместная работа над кодом и README.}
\end{document}
