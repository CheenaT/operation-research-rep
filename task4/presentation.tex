\documentclass{beamer}
\usepackage[utf8]{inputenc}
\usepackage[T1]{fontenc}
\usepackage[russian]{babel}

\usetheme{Antibes}
\usecolortheme{dolphin}

\usepackage{booktabs}
\usepackage{gensymb}

\title{Практикум на ЭВМ\\ Мечи залива работорговцев}

\author{Чистяков Иван \\ Раева Анастасия \\ Боданова Елена}

\institute{МГУ ВМК \\ Исследование Операций}

\date{2020 г.}

\begin{document}

    \begin{frame}
        \maketitle
    \end{frame}
    
    \begin{frame}{Постановка задачи}
        Два основных поставщика стали -- это WesterosInc и Harpy \& Co. 
        При заключении эксклюзивного договора на поставку каждая компания предлагает хорошую скидку.\\
        Неоходимо принять взвешенное рациональное решение вопроса о том, с какой из компаний следует заключить эксклюзивный договор на поставку стали.
        
    \end{frame}
    \begin{frame}{Исходные данные}
        Есть записи о производстве мечей каждым из кузнецов-безупречных, а также данные о количестве сломанных мечей в каждый из месяцев ведения боевых действий.
    \end{frame}
    \begin{frame}{Цель работы}
        Провести разведочный анализ данных и определить с кем выгоднее заключить договор.\\
        Разведочный анализ данных -- анализ основных свойств данных, нахождение в них общих закономерностей, распределений и аномалий, построение начальных моделей, зачастую с использованием инструментов визуализации.
        
    \end{frame}
    
    \begin{frame}{Анализ}
      Найденные метрики:
        \begin{itemize}
            \item Произведенные мечи 
            \item Количество сломанных мечей 
            \item Качество каждой поставки
            \item Плотность поломок
            \item Срок службы мечей
        \end{itemize}
    \end{frame}
    
    \begin{frame}{За все время}
        \begin{figure}[H]
             \centering
             \begin{subfigure}
                \includegraphics[width=5cm]{Presentation analysis/graph1.png}
             \end{subfigure}
            \begin{subfigure}
                \includegraphics[width=5cm]{Presentation analysis/graph2.png}
            \end{subfigure}
        \end{figure}
        \begin{table}[H]
            \begin{right}
                \begin{tabular}{|c|c|c|}
                    \hline
                    Поставщик & Произведено & Дефекты \\
                    \hline
                    Westeros Inc. & 31625 & 8268 \\
                    \hline
                    Harpy \& Co & 31523 & 6080 \\
                    \hline
                \end{tabular}
            \end{right}
            \\Произведено примерно одинаковое количество мечей, однако 
            число поломок у Westeros Inc. значительно больше.
        \end{table} 
        
        
    \end{frame}
    
    \begin{frame}{Дефектов к произведенному}
        \begin{figure}
            \centering
            \includegraphics[width=6cm]{Presentation analysis/graph3.png}
        \end{figure}
        Westeros.inc 0.26\\
        Harpy.co 0.19\\
        Наводит на мысль, что сталь Harpy\&co имеет качество лучше, чем Westeros inc.
    \end{frame}
    
    \begin{frame}{Анализ по месяцам}
        \begin{figure}
            \centering
            \includegraphics[width=8cm]{Presentation analysis/graph4.png}
        \end{figure}
        Видно, что поставки каждый месяц примерно одинаковые. Это дает
        нам возможность использовать данные, чтобы получить достоверную информацию.
    \end{frame}
    
    \begin{frame}{Анализ по месяцам}
        \begin{figure}
            \centering
            \includegraphics[width=8cm]{Presentation analysis/graph5.png}
        \end{figure}
       Первые два месяца значения близки, но потом качество Harpy становится значительно лучше в сравнении с Westeros.
    \end{frame}
    
    \begin{frame}{Качество партии}
        \begin{figure}
            \centering
            \includegraphics[width=8cm]{Presentation analysis/graph6.png}
        \end{figure}
        Со временем качество обеих компаний становится выше, но Harpy очевидно лидирует.
    \end{frame}
    
    \begin{frame}{Плотность поломок}
        \begin{figure}
            \centering
            \includegraphics[width=8cm]{Presentation analysis/graph7.png}
        \end{figure}
        Качество стали Harpy улучшилось относительно начального значения. 
        Мечи Westeros'a в среднем стали ломаться немного больше к шестому месяцу.
    \end{frame}
    
    \begin{frame}{Срок службы мечей}
        \begin{figure}
            \centering
            \includegraphics[width=8cm]{Presentation analysis/graph8.png}
        \end{figure}
        После первого месяца эксплуатации число поломок мечей Westeros'а почти в 5 раз превосходит Harpy.
        Заметные изменения происходят после 4 месяца эксплуатации. Мечи из стали Harpy ломаются чаще.
    \end{frame}
    \begin{frame}{Подведение итогов}
        Для сражений с запланированной длительностью меньше 4 месяцев отлично подходят мечи из стали Harpy \& Co. Их показатель поломок меньше почти в 5 раз. После наблюдается резкий скачок поломок, но при ежемесячных поставках это не является проблемой. 
        
    \end{frame}
\begin{frame}{Вывод}
      Качество мечей определяется качеством стали. Компания Harpy \& Сo лучше по показателям.
      Опираясь на проведенный анализ, делаем вывод, что заключать договр следует с Harpy \& Co.
        
    \end{frame}

\end{document}
